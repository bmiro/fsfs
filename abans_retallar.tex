\documentclass[a4paper,11pt]{scrartcl}
\usepackage[utf8x]{inputenc}
\usepackage[catalan]{babel}
% A ses llengües llatines, el primer paràgraf ha d'anar tabulat
\usepackage{float}
\usepackage{graphicx}
\usepackage{multirow}
\usepackage{hyperref}
\usepackage{url}

\graphicspath{{diagrames/}}

% Estil de seccions
%\titleformat{\section}{\large\sectfont}{\thesection}{1em}{}
%\titleformat{\subsection}{\bfseries\sectfont}{\thesubsection}{1em}{}
% Estil numeracio subseccions http://help-csli.stanford.edu/tex/latex-sections.shtml#number
%\def\thesubsection{\alph{subsection})}

%% Tamany del codi font python inserit amb el minted
\newcommand{\codeSize}{\footnotesize}

\title{Recensió STS: \\ \huge{Free Software Free Society} \\ \normalsize{Richard Mathew Stallman}}
\author{Bartomeu Miró Mateu \thanks{bartomeumiro a gmail punt com} \\}

\begin{document}

  \maketitle

  \newpage
  \setcounter{page}{2}
  \tableofcontents
  \newpage

  \section{Motivació de la lectura}
De tota la temàtica de l'assignatura la que personalment em sent més vinculant i de
la qual hem consider un militant es del programari lliure. Tal vegada hagués
estat interessant llegir un llibre d'alguna altre temàtica per tal de tocar
mes temes satisfent aquella cita:
  
  No llegeixis tot allò que creus.
  
De l'original

  No et creguis tot allò que llegeixes.
  
El cas és que \emph{Free Software, Free Society} es una lectura obligada
per qualsevol defensor del progrogramari lliure i així
es maten dos pardals d'un tir.

  \section{Dades bibliogràfiques}


autor, títol, lloc d'edició, editorial, colecció i any d'edició

  \section{L'autor Richard Mathew Stallman}
  
Richard Mathew Stallman o \emph{rms} es considerat el pare del programari lliure i
pioner del terme \emph{copyleft}.

Stallman nasqué al 1953 i es formà com a físic al MIT on treballà com a programador
fins que inicià el moviment del programari lliure amb el projecte GNU el qual
pretenia ser un sistema operatiu lliure basat en \emph{Unix}. 
Stallman és considerat un hacker de personalitat peculiar
ha entregat la seva vida al programari lliure convertint-se en gairebé un profeta de la causa,
cosa amb la qual ell sovint bromeja \footnote{TODO Referencia san ignucio}.

A banda de ser un reconegut programador per la comunitat també es conegut per el seu
difícil caràcter possiblement per el sindrome d'Asperger.

Finalment cal destacar que \emph{rms} es mostra molt compromés politicament declarant-se activista
solidaritzat amb un gran nombre de causes com l'ecologisme, socialisme o i en contra de mecanismes
de control per part dels més poderosos com ara els carnets d'identitat,
targetes de crèdit, xarxes socials i un llarg etc. Dit activsime el segueix de manera religiosa
i obsesiva la qual cosa el converteix en una persona molt coherent alhora que es criticat
per quedar al marge de segons quins aspectes de la societat.

  \section{El llibre}

El llibre és un recull estructurat d'assajos que ha anat publicant Richard M. Stallman
al llarg dels últims anys relacionats amb el programari lliure. Els assajos estan ordenats
seguit un fil argumental oferint una base conceptual i de motivacions per iniciar el moviment
del programari lliure fins a l'aprofundització de temes concrets com ara el \emph{copyright} o
les patents de programari.

En l'inici del llibre gran part dels assajos es corresponen
als primers en ordre cronològic ja que són els que parlen de les bases i l'inici històric
del moviment del programari lliure però sovint es romp introduint d'altres assajos
que perfilen millor els temes que es van tractant.

Dita ruptura temporal es gairebé invisible al lector, de fet l'única manera de sabre
quin és l'ordre cronològic dels assajos es cercant la data de publicació ja que el
contingut del text està escrit de manera que sembla atemporal, gran part de la
problemàtica i arguments no han canviat gairebé gens dels últims 20 anys fins ara.

S'ha de tenir amb compte que l'autor sempre s'ha mogut en un ambient tècnic i per
tant el \emph{boom} d'Internet, que ha estat el gran canvi en el sector dels últims 20-25
anys, gairebé no l'afecta ja que des de l'inici al seu entorn ja està completament
en xarxa i dona per suposat aquest mitjà.

El llibre està escrit perquè qualsevol persona en pugui llegir, entendre i assimilar
els arguments. No obstant el fet d'entendre com són els programes i haver tingut
experència com a programador facilita la comprensió de determinats temes. Un exemple
simple seria que un programador sap perfectement que desensamblar un programa
per tal de modificar-lo sense el codi font es una feina inviable, mentre que 
una una persona aliena al tema simplement s'ha d'agafar l'argument com a dogma.

Algun d'aquests temes són tractats a la introducció de manera breu encara
que com s'ha dit no es el mateix creurer-s'ho que haver-ho sofrit davant el teclat.

De totes maneres si s'enten la particularitat del programari que pot ser copiat sense
cost i com es desenvolupa tot el llibre es pot agafar com una divagació filosòfica
sobre això.

  \subsection{Estructura}

El llibre està dividit en quatre seccions on dins cada una hi ha un conjunt d'assajos
de temàtica relacionada. Dins cada assaig sovint hi ha una enumeració de punts o temes
que envolten dita qüestió clarament separats per capçaleres de títol.

La primera secció està dedicada a l'explicació del projecte GNU on s'exposen
els inicis del projecte i les motivacions personals per iniciar el moviment del
programari lliure, el projecte GNU.

La secció s'inicia amb un component històric per mostrar
els principis ètics del programari lliure i acaba immers de ple mostrant la
filosofia del programari lliure.

En la segona secció ja trobam tota una bateria d'assajos relacionats amb
copyright, copyleft i patents de programari que cerquen l'origen de cada terme
, explicant quin es l'esperit del terme a qui haurien de beneficiar, a qui
beneficient en realitat, a qui perjudiquen i com els podem emprar
al nostre favor o com haurien de ser modificats per beneficiar a la
societat i no només uns pocs. En funció de l'assaig s'agafa un terme o un altre
intentant desmentir tòpics i veure els prejudicis associats a cada terme.

En la tercera secció es torna a fer un repas de les idees exposades
als dos anteriors fent més emfasi en la influènica del programari
lliure en la societat intentant generalitzar encara més les
consequències del programari lliure ampliant la visió a la societat
en conjunt i no només com beneficia a cada un dels individus.

Finalment l'ultima secció està dedicada a les tres llicències més famoses
de la \emph{Free Software Foundation}, la \emph{GNU General Public License},
\emph{GNU Lesser General Public License} i \emph{Free Documentation License}.
En aquest apartat simplement estàn escrites les llicències sense introduccions ni comentaris.

Cal dir que les úniques seccions realment diferenciades són la primera on es veu
clarament com es parla dels inicis històrics i motivacions del projecte GNU i la quarta on hi ha les llicències.
Les demés seccions són dificils de distingir i hi ha assajos que tant podrien estar a una com l'altre.

  \subsection{Eixos argumentals}
  
En aquest punt s'intenten explicar els eixos argumentals del llibre, vindria a ser un intent
de síntesi del que queda en haver llegit el llibre.

\subsubsection{El programari una matèria particular}
El programari degut a la seva manera de ser, son bits informació, pot ser copiat sense cost.
A més el programari evoluciona i té sentit fer hi modificacions per millorar-lo i adaptar-lo
a les nostres necessitats. Així doncs qui tigui el codi font d'aquell programari te el 
poder de fer aquestes modificacions i adaptacions, a més de poder sabre que es el que
realment fa aquell programa. En contrast l'usuari del programa pot tenir sols la versió
binària o executable que no permet veure ni tocar les etranyes del programa.

Així doncs la persona que sols te un executable esta a la mercè de la volutat de qui
te el codi font. Si be es cert que l'usuari es aparentment lliure de triar el programa
que vol executar.

Aquest es el model que hem viscut tots sobre el programari uns el fan i els altres el 
consumeixen.

Aquest model no sempre ha estat així i de fet ha vingut imposat per les grans
companyies de programari. Inicialment els programes es compartien juntament
amb el codi font, de tal manera que qualsevol que tingues el programa també tenia
la possibilitat de modificar-lo i si volia seguir-lo compartint. Aquest es el model
en que Stallman va iniciar-se com a programador i al que vol tornar després
de la perversió del programari imposat per les empreses de programari.

La clau es que les empreses de programari cercen protecció de les lleis ja sigui
adaptant les ja existens per objectes materials com creant-ne d'espeficiques per ells.
Segons Stallman el que pretenen les grans empreses de programari es que primi el seu
benefici econòmic particular per sobre de la llibertat dels usuaris de compartir i
cooperar.

En el llibre es destaca que el programari lliure no es un tema de cost,
el programari lliure pot ser de pagament però no pots impedir que despres quin
t'ha pagat el programari el comparteixi.

El que preten fer veure Stallman es que no podem aplicar al programari
la mateixa lògica que els objectes materials. Els objectes materials si els
dones tu el perds mentre que amb el programari tu pots donar-lo sense
perdre'l.

Un programa per ser lliure segons RMS ha de donar a l'usuari quatre llibertats:
  
  Ha de poder execurar-se per qualsevol propòsit.
  Ha de poder ser redistribuit.
  Ha de poder ser modifiat.
  Ha den de poder ser redistribuides les seves modificacions.
  
Adicionalment Stallman creu que per evitar que aquest programa
deixi de ser lliure s'ha de afegir la condició de que les versions
modificades han d'estar sobre la mateixa llicència que l'original,
d'això s'en diu la condició vírica de la llicència.

Per tal que això tingui sentit qualsevol versió binaria del programa
que es entregada ha de venir amb una còpia del codi font o amb la
manera d'obtenir-lo.

\subsubsection{Les quatre llibertats del programari lliure}
Així doncs convé entrar en detall en les quatre llibertats mencionades
anteriorment ja que són els pilars del programari lliure i tota
la filosofia associada.

\paragraph{La primera llibertat} preten atacar les limitacions d'us del programa,
sonvint programes privatius permeten l'us sense cost si la
tasca es educacional o sense ànim de lucre però després s'imposen
restriccons en l'àmbit professional. Segons Stallman això no
es mes que una trampa per acostumar-se al programa i després
tenir-te una dependència. Així doncs el fet de permetre que en l'ambit
educacional sigui gratuit o es permeti el seu us no es un fet
altruista sinó completament interessat.

Aquest punt pot aixecar controverises certes controvèrsies. N'hi ha
una que no apareix al llibre però consider interessant. Suposem
un programador pacifista que elabora un programa i el llicencia
sote GNU GPL. Així doncs el programa pot ser emprat per qualsevol
propòsit. Suposem ara que algun exèrcit pot emprar aquest programa
o alguna versió lleguerament modificada per fins bèlics.

Al programador d'ideologia pacificsta tal vegada no li agradi aquest
fet i haurà de decidir que prima més per ell, si un us inapropiat del
programa o la llibertat global.

Stallman al llibre parla directament d'aquest problema però si que
s'intueix que per ell es molt mes important la llibertat, un programa
un cop lliurat l'usuari passa a tenir tots els drets tant per coses
bones com dolentes. Segons Stallman les coses \emph{dolentes}
s'han de controlar mitjançant altres eines legals que no pas la llicència,
que el fet de incloure limitacions d'ús en una llicència lliure
pot sembrar un mal precedent.

\paragraph{La segona llibertat} parla del dret a la redistribució per Stallman
això s'ha de veure com compartir. Compartir es bo i ens ho ensenyen
així des de petits però a mesura que ens feim grans veim que aquest
fet tant bonic sovint xoca amb altres interessos.

Per Stallman si algu empra un programa i el seu vei també el vol
o necessita se li ha de deixar, l'autor del programa no te cap pèrdua
amb que un altre usuari empri el seu programa encara que l'autor
el faci pagar. Si el programa es lliure i qui el necessiti pot
asumir el preu no te cap motiu per no pagar, simplement ho fara
perque sap que aixi contribueix al desenvolupament del mateix.
Per altra banda si no el paga tal vegada sigui perque el cost
es massa elevat, en aquest cas ens trobam que aquesta persona
veu limitades les seves accions per el captial que posseeix,
que emprar programari es torna una cosa elitista i per tant injusta.

Aqui hom pot pensar que així els autors o programadors no
poden guanyar diners o viure del seu treball, això serà detallat
en un punt posterior on es parla del model de negoci.

\paragraph{La tercera llibertat} és conseqüència de que
un cop un programa pot ser executat i redistribuit te sentit que 
sigui modificat. Tot programa pot ser modificat millorat, adaptat
etc. Com a informàtics sabem que no existeixen els programes
sense errors, així doncs te sentit que podem modificar els nostres programes.

Aquesta llibertat el que fa es llevar-te la dependencia el programador 
que inicialment ha fet el programa permetent que tu mateix puguis
fer les modificacions o comanar-ho a algu que ho faci.

Segons Stallman els autors programari empren simils amb la realitat només
quan els convé, per exemple diguen que el programari es un producte i que
per tant s'han de pagar les còpies. Per altre banda si es fa un simil
amb el món material en les modificacions del programa trobam que
un obrer pot construir-te una casa però després no tens perque contractar
el mateix obrer per fer modificacions o remodelacions. Amb el programari
si aquest no es lliure quedes vinculat per llicència o simplement
perque no disposes del codi font a l'autor que l'ha fet.

Així doncs que el programa pugui ser modificat es el més \emph{natural}
l'artifici es limitar aquesta llibertat per benficiar a uns pocs.
Entenent natural com el fet més comú i que al llarg de la història
han anat perdurant.

Aquí es pot posar l'exemple del Windows XP el qual no estava en català.
La Generalitat va creure que era positiu per la llengua catalana
que aquest fos traduit. Per llicència dita traducció no pot fer-se
sense el consentiment de Microsoft, així doncs la Generalitat
va haver de pagar per tenir el consentiment per fer la traduccció.
A més la Generalitat va pagar també a un tercer que va fer la traducció
i finalment aquesta va ser distribuida sense cost.
Microsoft va obtenir un millor producte sense esforç i a més
cobrant per senzillament \emph{donar el consentiment}.

En aquest exemple es veu clara la dependència que pot generar un programa
no lliure que a més ha aconseguit una posició dominant i l'absurd
de pagar a algú  per millorar el seu propi producte que també el beneficia.

Hom pot preguntar: I que passa si no es tenen diners per pagar el
consentiment del Microsoft per traduir-lo al català?
No tenim dret els catalans a poder veure el sistema operatiu
en la nostra llengua? Encara que nosaltres facem la feina?

\paragraph{Finalment la quarta llibertat} parla de poder redistribuir les modificacions,
aquesta si s'han assumit les primers es lògica, l'unic argument
que pot formular-se en contra es que la versió modificada del programa
sigui millor que l'inicial i que així doncs l'original deixi de ser
emprat. Dit argument sols perjudica a l'autor de l'original si 
es que aquest te un sistema de retribucions per numero de copies del programa
en us. Per altra banda tenim tots els usuaris emprant una versió millor
del programa. Altre cop cop ens trobam en l'enfrontament autor usuari
que ja ha estat discutit en els punts anteirors.


\subsection{Codi obert \emph{(Open Source)} vs Programari Lliure}
Segons Stallman la diferència crucial es que el programari lliure exigeix
que les copies també siguin lliures cosa que el codi obert no.

L'unic motiu a favor del codi obert es incitar algú a
iniciar-se en el codi obert i després tancar el codi o tenir part del
codi tancada i l'altre oberta.

Aquest fet es perjudicial ja que no elimina el problema, un com s'ha
tancant el codi o part d'ell encara que aquest fos inicialment lliure tornam
a caure en exactament els mateixos problemes que amb codi privatiu.

Adicionalment tot el treball que pot haver fet una comunitat per un programa
si aquest es lliure tota modificació torna a revertir a la cominitat,
si aquest es sols obert algú pot fer una millora que és distribuida 
de manera no lliure i que per tant no torna a revertir a la comunitat.

Per Stallman la questió no es nomes poder veure el codi si aquest no pot
ser modificat no te cap sentit. A més Stallman creu que el fet de dir
Open Source dona un valor afegit al teu programa, aquest valor afegit
fa creure que pots modificar el programa però no te perque ser aixi,
una llicència open source pot sols permetret veure el codi, o veure i modificar
però no redistribuir etc.

Així doncs Stallman creu que el codi obert es beneficia del concepte de programari
lliure però essent una trampa ja que no tenen perque ser equivalents.

Posem un exemple de dit cas. Suposem que feim un programa de compresió de dades.
Ens interessa molt que la gent empri el nostre format així doncs feim
que l'algorisme de compresió sigui open source.

D'aquesta manera aconseguim que molta gent vegi el codi i fins i tot l'implementi
als seus sistemes operatius. D'aquesta manera aconseguim que el nostre format
s'extengui. Un cop està establert el que feim es tancar el codi i fer unes petites
millores. En aquest punt per inèrcia la gent seguirà emprant el format però
com que no tenen dret a modificació del codi font original amb open source
nomes el poden veure i compilar hauran d'abandonar-lo per la versió privativa
i ja tornam a estar amb el problema inicial.

El codi obert no garanteix la llibertat de l'usuari, de fet segons Stallman
pot ser emprat de trampa comercial. La diferencia entre programari lliure
i open source es precisament el caràcter víric que assegura que el programa
no serà tancat així doncs si volem un programa lliure i no només actualment
sinó també en un futur s'han d'evitar les llicències open source.


\subsection{Programari lliure i societat}
El fet de emprar programari lliure el que fa es fomentar el valor de la cooperació,
tots participam en el desenvolupament del programa; en front de la competència,
tots desenvolupam programes diferents i el millor tendrà més èxit.

Si es mira desde el punt de vista de l'eficiència dos programes que fan el
mateix i son incompatibles (el programari privatiu tendeix a fer-se incompatible
amb altres perque sols sigui emprat ell) arribarà un moment en que possiblement
un dels dos sigui més emprat fins al punt que el segon deixerà de tenir sentit.

El programa que deixa de tenir sentit també es fruit de moltes hores de treball
i aquestes quan el programa desapareix també desapareixen. Es cert que tal
vegada la competència hagi fet que les funcionalitats d'aques programa
hagin induit a millorar l'altre que també les ha implementat però dites
funcionalitats hauran estat implementades per duplicat.

Stallman creu que si en lloc d'això es coopera en un mateix programari el nombre
d'hores totals invertides es menor ja que no hi ha hores tudades fent centenars
de programes iguals com passa en el cas del programari privatiu.

El fet que hi hagi programes que fan el mateix no es dolent de per si,
el que es dolent que tu per voler afegir una funcionalitat a un programa
l'hagis de fer des de zero perquè no tens accés al codi font.

Per altra banda al poder ser el programa redistribuit els usuaris el
poden compartir en funció de les seves necessitats així doncs creen
vincles amistosos entre ells. Dins vincles no poden ser establerts
amb el programari privatiu ja que els usuaris han de traïr les seves
amistats per obeir al desenvolupador que diu que està prohibit redistribuir-se les còpies.

Així doncs el programari lliure fomenta valors com la cooperació, eficàcia i la solideritat,
tots ells valors que la societat classifica com a positius i que han de ser potenciats.

\subsection{Programari lliure i el seu model de negoci}

En el llibre Stallman de manera incansable repeteix que no s'ha de confondre
lliure i gratuït ja que en anglès \emph{free} pot voler dir ambdues coses.

De fet en diverses ocasions remarca que va iniciar el moviment del programari lliure
venent els seus programes que eren evidentment lliures.

Stallman reconeix que amb el programari lliure tal vegada no es guanyin tants tants
diners com el privatiu (recordem que un dels homes més rics del món s'hi ha fet 
venent programari) però es això un problema? Segons Stallman clar que no, que uns
molt pocs desenvolupadors o empresaris no siguin multimilionaris en front
de que tots els usuaris siguin més lliures no suposa cap problema en absolut.

Així i tot hi ha diversos punts a tractar sobre el cost del programari
emmarcats en el sistema econòmic en el que vivim.

En primer lloc pot pensar-se que si no es paga per les còpies els
desenvolupadors o autors no tendran benefici.

Per començar aquest argument es fals, parteix del fet que ningú
pagarà cap copia i no és cert. Si la gent valora el programa
no hi ha motius per pensar que voldan copiar-lo sense pagar i si
ho fan possiblement es perquè no puguin pagar el preu fixat.

De fet hi ha una experència actual i no contemplada al llibre i 
es que en el mon dels videojocs existeix el Indie Pack, aquest
son un conjunt de videojocs on la gent
paga el que vol i pot redistribuir-los gratuitament.

Per ara totes les edicions (i ja van 5) abans de la primera
setmana ja havien passat el milió de dòlars en recaptació.

Una altre manera de pagar als desenvolupadors es entenent
el programari com un servei i no com un producte. Així doncs
tu a mes (o en lloc) de pagar la còpia pages al desenvolupador
per les modificacions al programa, perque l'adapti al que tu vols
o simplement continui amb el desenvolupament, com una quota.

Per altra banda com que els programes podens ser emprats per
tothom també tindria sentit que els desenvolupadors rebessin subvecions.

De fet si es paguessin a programadors en lloc de comprar programari
privatiu per part de l'administració públic la cosa ascendeix
a xifres astronòmiques. Tant es aixi que a comunitats españoles
com Extremadura prefereixen emprar programari lliure i estalviar
diners a emprar el programari privatiu.

També pot considerar-se el fet que els estudiants universitaris
vinculats a carreres tècniques poden realitzat practiques i PFC
relacionats amb programari lliure.

Finalment Stallman també menciona la possibilitat que cada
equip informàtic venut tengui una petita taxa que gestioni
l'estat o alguna entitat per tal de repartir-la entre
desenvolupadors de programari lliure.

Cal esmentar que determinades empreses que no es dediquen
al programari però si l'empren poden argumentar que si
alliberen el seu codi estan afavorint a la competència.

En aquest punt es pot dir que si es així la competència
també voldra millorar el programa i per tant si aquest
es lliure les millores tornen a revertir a l'empresa original.

Per altre banda si bé es cert que tal vegada una empresa pugui
ser perjudicada s'ha de tenir amb compte la vist en conjunt
la socientat en general es veu beneficiada. Ja comença a ser
hora de veure que vivim en un mon globalitzat i no hem 
de mirar nomes el benefici propi o de curt abanst sinó
el que ens afecta a tots.

\subsection{El problema del nom i paraules prohibides}
Segons Stallman el llenguatge es una arma molt perillosa
i sovint emprar segons quines paraules associa el fet a
determinats conceptes que a priori poden no ser clars,
es per això que dedica un assaig sencer a paraules
mostrant els conceptes que porten associats i com
determinats sectors les fan servir per manipular 
l'usuari en el seu benfici.

Crec que es interessant mencionar-ne unes quantes
ja que despres d'entendre la situació si es sent
un interlocutor rapidament es pot veure quina
es la seva manera de pensar i sobretot es poden
veure quins pensaments hi ha sovint ocults darrera
les seves paraules.

Per altra banda veient els conceptes associats a les paraules
també s'agafa consciència de com funciona el món en el que
vivim i quin es l'origen de cada terme i com a conseqüència
de cada situació que vivim.





\subsection{Quina es la batalla}

Stallman no parla de prohibir el programari privatiu, simplement recomana
no emprar-lo, intenta fer veure que si ningú l'empra i tothom empra
programari lliure les coses funcionaran millor. El problema es que
a dia d'avui el programari lliure es perseguit i posat en perill per coses
com les patents de programari, a més de no comptar amb els suport
de grans fabricants.

Al cap i a la fi per la major part de la gent la informàtica es una
eina que ha de funcionar, així doncs a la gent li costa emprar una cosa
amb la qual te un suport tècnic baix o difícil en front d'una empresa
que tot hi no semblar molt bona dona una seguretat.

Per tant el camí es desenvolupar bon programari lliure i conscienciar a
la gent dels aventatges que té front a la problemàtica del programari
privatiu, així com lluitar perquè lleis com les patents de programari
no s'intaurin en la legislació o siguin derogades.









  \section{Valoració personal}

L'estructura del llibre pot millorar-se. S'ha de tenir amb compte que es simplement
un recull d'assajos però a l'acabar-lo i pensar en un assaig es dificil ubicar-lo en alguna de la seccions.

En aquest sentit si hom espera un llibre amb un fil argumental completament definit no ho trobarà,
un cop llegit els primers assajos s'enten la idea del programari lliure i després senzillament
es fan voltes desde mútliples vessants i veien les conseqüències en inumerables àmbits i situacions.

S'ha de dir que sovint al plantejar els diferents àmbits l'autor també mostra la seva ideologia
 la qual cosa també enriqueix al lector. Algun cas concret
es del capitol 4 de la primera secció al paràgraf \emph{The Law} on es fa una petita reflexió
de que la gent no ha de limitar les seves accions al que dicta la llei ja que aquesta es va format
i depen del context cultural. Explicitament cita lleis racistes dels Estats Units que per ser llei
no la feien menys racista i injusta.

Com aquesta en multiples ocasions Stallman detalla el funcionament d'algun element de la societat
que sovint es veu com inquestionable mostrant el seu origein i sovint com aquest s'ha pervertit
per afavorir els més poderosos.

Dins del llibre també es poden observar petits detalls personals de l'autor. Un que va soprendrem
i alhora dibuixar un somriure es el fet que parli d'un dels seus amors que es deia (o portava el pseudonim)
d'Alyx i que com a bon informàtic enamporat Stallman li dedicà una modul del kernel Hurd posant-li el seu nom.
Desafortunadament i com sovint bromeja en públic la cosa no funcionà i defensar el programari lliure
a mode de profeta no requereix celibat però sovint l'implica.

Per altra banda hi ha d'altes qüestions que ja depenen més de la percepció del lector. En aquest cas
vaig notar certa rabia cap al moviment Open Source al capitol \emph{Why Free Software is better than Open Source}.
En ell es percep cert recor cap a moviment \emph{Open Source} per haver eclipsat part del seu treball. Dit odi no es veu
cap al programari privatiu el qual es l'enemic, simplement es plateja com una molt mala opció perjudicial però
l'argumentació es neta. En canvi en l'enfrontament amb l'\emph{Open Source} s'exposen els arguments i s'afegeix cert
tò d'enveja que per altre banda crec innecessari i menys d'una persona que està acostumada a viure a contra corrent
sense rebre gaire reconeixement social.

El llibre en si està prou bé es molt complet però pel meu gust un xic repetitiu. Tal vegada s'en podria
fer una versió mes reduïda amb un subconjunt dels assajos que apareixen i s'en treuria més rendiment,
seria més ràpid i mes fàcil de difondre al públic en general. Per una banda no vull desmereixer l'autor
però per l'altre he de reconeixer que tot hi estar molt interessant en el tema alguns dels capitols
s'han fet dificils de llegir tot i ser breus. Si be es cert que eren del final quan ja s'ha exposat
gairebé tot i simplement es recargolen més els argumans a més de que per part meva que no estic acosumat
a llegir llibres d'assajos.

  \subsection{Assajos interessants}
Completant la secció anterior aqui expòs quins son els assajos mes recomanables i el perquè.

En general els mes recomanats son els del principi ja que es on s'assenten les bases i a més
hi ha mes component històric que a més de ser interessant resula més fàcil de llegir.





   \let\thefootnote\relax\footnotetext{
 Aquest document està baix llicència \href{http://creativecommons.org/licenses/by-sa/3.0/}{Creative Commons Atributive Share-Alike 3.0}
 per tant es pot compartir, modificar i distribuir, però citant l'autor original i sense modificar la llicència.\bigskip}
\let\thefootnote\relax\footnotetext{El document en versió digital i el codi font el trobareu a \\
\url{https://github.com/bmiro/fsfs}\bigskip}
\let\thefootnote\relax\footnotetext{Aquest document ha estat desenvolupat emprant programari lliure:}
\let\thefootnote\relax\footnotetext{\href{http://www.tug.org/applications/pdftex/}{\LaTeX} i \href{http://www.tug.org/applications/pdftex/}{Kile} per el text.
}

\let\thefootnote\relax\footnotetext{
\begin{center}
\begin{tabular}{cc}
\includegraphics[height=35pt,keepaspectratio=true]{diagrames/by-sa.png}
 & \includegraphics[height=35pt,keepaspectratio=true]{diagrames/gnu.png}
\end{tabular}
\end{center}
\begin{center}
\begin{tabular}{cc}
 \includegraphics[height=35pt,keepaspectratio=true]{diagrames/latex.png}
 & \includegraphics[height=35pt,keepaspectratio=true]{diagrames/kile.png}
\end{tabular}
\end{center}
}


\end{document}




