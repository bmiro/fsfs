\documentclass[a4paper,10pt]{scrartcl}
\usepackage[utf8x]{inputenc}
\usepackage[catalan]{babel}
% A ses llengües llatines, el primer paràgraf ha d'anar tabulat
\usepackage{float}
\usepackage{graphicx}
\usepackage{multirow}
\usepackage{hyperref}
\usepackage{url}

\graphicspath{{diagrames/}}

\title{\Large{Recensió STS:} \\ \huge{Free Software Free Society} \\ \normalsize{Richard Mathew Stallman}}
\author{Bartomeu Miró Mateu \thanks{bartomeumiro a gmail punt com} \\}

\begin{document}

  \maketitle

  \section{Motivació de la lectura}
De tota la temàtica de l'assignatura la que personalment em sent més vinculant i de
la qual hem consider un militant es del programari lliure. Tal vegada hagués
estat interessant llegir un llibre d'alguna altre temàtica per tal de tocar
més temes satisfent aquella cita:

\begin{quotation}
  No llegeixis tot allò que creus.
\end{quotation}

De l'original

\begin{quotation}
  No et creguis tot allò que llegeixes.
\end{quotation} 
 
El cas és que \emph{Free Software, Free Society} és una lectura obligada
per qualsevol defensor del programari lliure i així es maten dos pardals
d'un tir.

  \section{Dades bibliogràfiques}

  Richard M. Stallman, Free Software Free Society, SoHo Books, 2002
  
  El mateix llibre està disponible baix llicència lliure GNU FDL i pot ser descarregat aquí:
  \url{www.gnu.org/philosophy/fsfs/rms-essays.pdf}
  
  D'aquest llibre n'existeix una segona edició de 2010 que conté nous assajos.

  \section{L'autor Richard Mathew Stallman}
  
  Richard Mathew Stallman o \emph{rms} és considerat el pare del programari lliure i
pioner del terme \emph{copyleft}.

  Stallman nasqué al 1953 i es formà com a físic al MIT on treballà com a programador
fins que inicià el moviment del programari lliure amb el projecte GNU el qual
pretenia ser un sistema operatiu lliure basat en \emph{Unix}.

  Stallman és considerat un hacker de personalitat peculiar
ha entregat la seva vida al programari lliure convertint-se en gairebé un profeta de la causa,
cosa amb la qual ell sovint bromeja auto denominant-se \emph{Sant iGNUcio}.

  Finalment cal destacar que \emph{rms} es mostra molt compromès políticament declarant-se activista
solidaritzat amb un gran nombre de causes com l'ecologisme, socialisme o i en contra de mecanismes
de control per part dels més poderosos com ara els carnets d'identitat,
targetes de crèdit, xarxes socials i un llarg etc. Dit activisme el segueix de manera religiosa
i obsessiva la qual cosa el converteix en una persona molt coherent alhora que es criticat
per quedar al marge de segons quins aspectes de la societat. Dita critica sumada
al possible síndrome d'Asperger (un autisme lleuger) fan que tingui una personalitat molt
difícil al tracte.


\section{El llibre}

  El llibre és un recull estructurat d'assajos que ha anat publicant Richard M. Stallman
al llarg dels últims anys relacionats amb el programari lliure. Els assajos estan ordenats
seguit un fil argumental oferint una base conceptual i de motivacions per iniciar el moviment
del programari lliure fins a l'aprofundiment de temes concrets com ara el \emph{copyright} o
les patents de programari.

  En l'inici del llibre gran part dels assajos es corresponen
als primers en ordre cronològic ja que són els que parlen de les bases i l'inici històric
del moviment del programari lliure però un cop superats els moments inicials els assajos
no tenen relació d'ordre temporal i de fet per l'estructura tampoc la necessiten.

  Dita ruptura temporal és gairebé invisible al lector, de fet l'única manera de sabre
quin és l'ordre cronològic dels assajos es cercant la data de publicació ja que el
contingut del text està escrit de manera que sembla intemporal, gran part de la
problemàtica i arguments no han canviat gairebé gens dels últims 20 anys fins ara.
S'ha de tenir amb compte que l'autor sempre s'ha mogut en un ambient tècnic i per
tant el \emph{boom} d'Internet, que ha estat el gran canvi en el sector dels últims 20-25
anys, gairebé no l'afecta ja que des de l'inici al seu entorn ja està completament
en xarxa i dona per suposat aquest mitjà.

  El llibre està escrit perquè qualsevol persona en pugui llegir, entendre i assimilar
els arguments. No obstant el fet d'entendre com són els programes i haver tingut
experiència com a programador facilita la comprensió de determinats temes. Un exemple
simple seria que un programador sap perfectament que des-assemblar un programa
per tal de modificar-lo sense el codi font es una feina inviable, mentre que 
una una persona aliena al tema simplement s'ha d'agafar l'argument com a dogma.
  Algun d'aquests temes són tractats a la introducció de manera breu encara
que com s'ha dit no es el mateix creurer-s'ho que haver-ho patit davant el teclat.
  De totes maneres si s'entén la particularitat del programari que pot ser copiat sense
cost i com es desenvolupa tot el llibre es pot agafar com una divagació filosòfica
sobre això.

  \subsection{Estructura}
  El llibre està dividit en quatre seccions on dins cada una hi ha un conjunt d'assajos
de temàtica relacionada. Dins cada assaig sovint hi ha una enumeració de punts o temes
que envolten dita qüestió clarament separats per capçaleres de títol en negreta.

\paragraph{La primera secció} està dedicada a l'explicació del projecte GNU on s'exposen
els inicis del projecte i les motivacions personals per iniciar el moviment del
programari lliure, el projecte GNU. Aquest inici serveix per mostrar
els principis ètics del programari lliure i immers de ple mostrant la
filosofia del programari lliure.

\paragraph{En la segona secció} ja trobam tota una bateria d'assajos relacionats amb
\emph{copyright}, \emph{copyleft} i patents de programari que cerquen l'origen de cada terme
, explicant quin es l'esperit del terme a qui haurien de beneficiar, a qui
beneficia en realitat, a qui perjudiquen i com els podem emprar
al nostre favor o com haurien de ser modificats per beneficiar a la
societat i no només uns pocs. En funció de l'assaig s'agafa un terme o un altre
intentant desmentir tòpics i veure els prejudicis associats a cada terme.

\paragraph{En la tercera secció} es torna a fer un repas de les idees exposades
als dos anteriors fent més èmfasi en la influència del programari
lliure en la societat intentant generalitzar encara més les
conseqüències de l'us massiu de programari lliure ampliant la visió
a la societat en conjunt i veient els beneficis no son sol qualitatius
(en nombre d'individus més lliures) sinó qualitatius.

\paragraph{Finalment l'última secció} està dedicada a les tres llicències més famoses
de la \emph{Free Software Foundation}, la \emph{GNU General Public License},
\emph{GNU Lesser General Public License} i \emph{Free Documentation License}.
En aquest apartat simplement estan escrites les llicències sense introduccions ni comentaris.

Cal dir que les úniques seccions realment diferenciades són la primera on es veu
clarament com es parla dels inicis històrics i motivacions del projecte GNU i la quarta on hi ha les llicències.
Les demés seccions són difícils de distingir i hi ha assajos que tant podrien estar a una com l'altre.

\subsection{Eixos argumentals}
  
En aquest punt s'intenten explicar els eixos argumentals del llibre, vindria a ser un intent
de síntesi del que queda en haver llegit el llibre.

\subsubsection{El programari una matèria particular}
El programari degut a la seva manera de ser, són bits informació, pot ser copiat sense cost.
A més el programari evoluciona i té sentit fer hi modificacions per millorar-lo i adaptar-lo
a les nostres necessitats. Així doncs qui tingui el codi font d'aquell programari té el 
poder de fer aquestes modificacions i adaptacions, a més de poder sabre que és el que
realment fa aquell programa. En contrast l'usuari del programa pot tenir sols la versió
binària o executable que no permet veure ni tocar les entranyes del programa.

Així doncs la persona que sols te un executable esta a la mercè de la voluntat de qui
te el codi font. Si bé és cert que l'usuari és aparentment lliure de triar el programa
que vol executar.

Aquest és el model que hem viscut sobre el programari: uns el fan i els altres el 
consumeixen.

Aquest model no sempre ha estat així i de fet ha vingut imposat per les grans
companyies de programari. Inicialment els programes es compartien juntament
amb el codi font, de tal manera que qualsevol que tingues el programa també tenia
la possibilitat de modificar-lo i si volia seguir-lo compartint. Aquest es el model
en que Stallman va iniciar-se com a programador i al que vol tornar després
de la perversió imposada per les empreses de programari.

En el llibre és destaca que el programari lliure no es un tema de cost,
el programari lliure pot ser de pagament però no pots impedir que després quin
t'ha pagat el programari el comparteixi.

El que pretén fer veure Stallman es que no podem aplicar al programari
la mateixa lògica que els objectes materials. Els objectes materials si els
dones tu el perds mentre que amb el programari tu pots donar-lo sense
perdre'l.

Un programa per ser lliure segons RMS ha de donar a l'usuari quatre llibertats:

\begin{enumerate}
  \item Ha de poder executar-se per qualsevol propòsit.
  \item Ha de poder ser redistribuït.
  \item Ha de poder ser modificat.
  \item Ha den de poder ser redistribuïdes les seves modificacions.
\end{enumerate}
  
Addicional-ment Stallman creu que per evitar que aquest programa
deixi de ser lliure s'ha de afegir la condició de que les versions
modificades han d'estar sobre la mateixa llicència que l'original,
d'això s'en diu la condició vírica de la llicència.

Per tal que això tingui sentit qualsevol versió binaria del programa
que es entregada ha de venir amb una còpia del codi font o amb la
manera d'obtenir-lo.

\subsubsection{Les quatre llibertats del programari lliure}
Així doncs convé entrar en detall en les quatre llibertats mencionades
anteriorment ja que són els pilars del programari lliure i tota
la filosofia associada.

\paragraph{La primera llibertat} pretén atacar les limitacions d'ús del programa,
per Stallman és molt més important la llibertat, un programa
un cop lliurat l'usuari passa a tenir tots els drets tant per coses
bones com dolentes. Segons Stallman les coses \emph{dolentes}
s'han de controlar mitjançant altres eines legals que no pas la llicència,
que el fet de incloure limitacions d'ús en una llicència lliure
pot sembrar un mal precedent.

\paragraph{La segona llibertat} parla del dret a la redistribució
que s'ha de veure com tenir la llibertat de compartir. Compartir és bò
i ens ho ensenyen així des de petits però a mesura que ens feim grans
veim que aquest fet tant bonic sovint xoca amb altres interessos.

Per Stallman si algú empra un programa i el seu veí també el vol
o necessita se li ha de deixar, l'autor del programa no te cap pèrdua
amb que un altre usuari empri el seu programa encara que l'autor
el faci pagar. Si el programa és lliure i qui el necessiti pot
assumir el preu no te cap motiu per no pagar, simplement ho farà
perquè sap que així contribueix al desenvolupament del mateix.

Per altra banda si no el paga tal vegada sigui perquè el cost
és massa elevat, en aquest cas ens trobam que aquesta persona
veu limitades les seves accions per el capital que posseeix,
que emprar programari és torna una cosa elitista i per tant injusta.

\paragraph{La tercera llibertat} és conseqüència de que
un cop un programa pot ser executat i redistribuït té sentit que 
sigui modificat. Tot programa pot ser modificat millorat, adaptat
etc.

Aquesta llibertat el que fa es llevar-te la dependència el programador 
que inicialment ha fet el programa permetent que tu mateix puguis
fer les modificacions o comanar-ho a algú que ho faci.

Un obrer pot construir-te una casa però després no tens perquè contractar
el mateix obrer per fer modificacions o remodelacions. Amb el programari
si aquest no es lliure quedes vinculat per llicència o simplement
perquè no disposes del codi font a desenvolupador que l'ha fet.

Així doncs que el programa que pugui ser modificat és la més \emph{natural}
l'artifici és limitar aquesta llibertat per beneficiar a uns pocs.
Entenent natural com el fet més comú i que al llarg de la història
han anat perdurant.

\paragraph{Finalment la quarta llibertat} parla de poder redistribuir les modificacions,
aquesta si s'han assumit les primers és lògica, l'únic argument
que pot formular-se en contra és que la versió modificada del programa
sigui millor que l'inicial i que així doncs l'original deixi de ser
emprat. Dit argument sols perjudica a l'autor de l'original si 
és que aquest té un sistema de retribucions per número de copies del programa
en ús. Per altra banda tenim tots els usuaris emprant una versió millor
del programa. Altre cop ens trobam en l'enfrontament autor usuari
que ja ha estat discutit en els punts anteriors.

\subsection{Codi obert \emph{(Open Source)} vs Programari Lliure}
Segons Stallman la diferència crucial és que el programari lliure exigeix
que les copies també siguin lliures cosa que el codi obert no.

L'únic motiu a favor del codi obert és incitar algú a
iniciar-se en el codi obert i després tancar el codi o tenir part del
codi tancada i l'altre oberta.

Addicional-ment tot el treball que pot haver fet una comunitat per un programa
si aquest es lliure tota modificació torna a revertir a la comunitat,
si aquest es sols obert algú pot fer una millora que és distribuïda 
de manera no lliure i que per tant no torna a revertir a la comunitat.

El codi obert no garanteix la llibertat de l'usuari, de fet segons Stallman
pot ser emprat de trampa comercial. La diferència entre programari lliure
i codi obert és precisament el caràcter víric que assegura que el programa
no serà tancat així doncs si volem un programa lliure i no només actualment
sinó també en un futur s'han d'evitar les llicències codi obert.

\subsection{Programari lliure i societat}
El fet de emprar programari lliure el que fa es fomentar el valor de la cooperació,
tots participam en el desenvolupament del programa; en front de la competència,
tots desenvolupam programes diferents i el millor tendrà més èxit.

Es cert que tal vegada la competència hagi fet que les funcionalitats de cert programa
hagin induït a millorar l'altre que també les ha implementat però dites
funcionalitats hauran estat implementades per duplicat.

Stallman creu que si en lloc d'això és coopera en un mateix programari el nombre
d'hores totals invertides és menor ja que no hi ha hores malgastades fent centenars
de programes iguals com passa en el cas del programari privatiu.

Per altra banda al poder ser el programa redistribuït els usuaris el
poden compartir en funció de les seves necessitats així doncs creen
vincles amistosos entre ells. Dins vincles no poden ser establerts
amb el programari privatiu ja que els usuaris han de trair les seves
amistats per obeir al desenvolupador que diu que està prohibit redistribuir-se les còpies.

Així doncs el programari lliure fomenta valors com la cooperació, eficàcia i la solidaritat,
tots ells valors que la societat classifica com a positius i que han de ser potenciats.

\subsection{Programari lliure i el seu model de negoci}
En el llibre Stallman de manera incansable repeteix que no s'ha de confondre
lliure i gratuït ja que en anglès \emph{free} pot voler dir ambdues coses.

De fet en diverses ocasions remarca que va iniciar el moviment del programari lliure
venent els seus programes que eren evidentment lliures.

Stallman reconeix que amb el programari lliure tal vegada no es guanyin tants
diners com el privatiu (recordem que un dels homes més rics del món s'hi ha fet 
venent programari) però es això un problema? Segons Stallman clar que no, que uns
molt pocs desenvolupadors o empresaris no siguin multimilionaris en front
de que tots els usuaris siguin més lliures no suposa cap problema en absolut.

Per desmentir que no es puguin guanyar diners o que el desenvolupament
de programari en un model de programari lliure pot pagar els desenvolupadors
Stallman proposa alguns sistemes de retribució.

La manera obvia es que es paga per cada copia que es distribueix
permetent que els usuaris si volen s'intercanviïn copies de les quals
no se n'obté dit benefici.

Una altre manera de pagar als desenvolupadors es entenent
el programari com un servei i no com un producte. Així doncs
tu a més (o en lloc) de pagar la còpia pages al desenvolupador
per les modificacions al programa, perquè l'adapti al que tu vols
o simplement continuï amb el desenvolupament, com una quota.

De fet si es paguessin a programadors en lloc de comprar programari
privatiu per part de l'administració públic la cosa ascendeix
a xifres astronòmiques, per tant les subvencions estatals 
no són una altre via de finançament.

Finalment Stallman també menciona la possibilitat que cada
equip informàtic venut tengui una petita taxa gestionada per
alguna entitat per tal de repartir-la entre
desenvolupadors de programari lliure.

  \section{Valoració personal}
  
  Donat que en el punt anterior ja he mostrat la síntesi argumental que crec haver extret del llibre
en aquest punt faig un comentari mes abstracte destacant detalls molt més puntuals.

En quan a l'estructura del llibre crec que pot millorar-se. S'ha de tenir amb compte que es simplement
un recull d'assajos però a l'acabar-lo i pensar en un assaig es difícil ubicar-lo en alguna de la seccions.

Un aspecte molt interessant es que sovint al plantejar els diferents àmbits l'autor també mostra la seva ideologia
 a qual cosa també enriqueix al lector. Algun cas concret
es del capítol 4 de la primera secció al paràgraf \emph{The Law} on és fa una petita reflexió
de que la gent no ha de limitar les seves accions al que dicta la llei ja que aquesta es va format
i depèn del context cultural. Explícitament cita lleis racistes dels Estats Units que per ser llei
no la feien menys racista i injusta.

Com aquesta en múltiples ocasions Stallman detalla el funcionament d'algun element de la societat
que sovint és veu com inqüestionable mostrant el seu origen i sovint com aquest s'ha pervertit
per afavorir els més poderosos.

Dins del llibre també es poden observar petits detalls personals de l'autor. Un fet que em sorprengué
i alhora dibuixar un somriure es el fet que parli d'un dels seus amors que es deia (o portava el pseudònim)
d'Alyx i que com a bon informàtic enamorat Stallman li dedicà una mòdul del \emph{kernel Hurd} posant-li el seu nom.
Des-afortunadament i com sovint bromeja en públic la cosa no funcionà i defensar el programari lliure
a mode de profeta no requereix celibat però sovint l'implica.

Per altra banda hi ha d'altes qüestions que ja depenen més de la percepció del lector. En aquest cas
vaig notar certa ràbia cap al moviment \emph{Open Source} al capítol \emph{Why Free Software is better than Open Source}.
En ell es percep cert rancor cap a moviment \emph{Open Source} per haver eclipsat part del seu treball. Dit odi no es veu
Cap al programari privatiu el qual és l'enemic, simplement és plateja com una molt mala opció perjudicial però
l'argumentació es neta. En canvi en l'enfrontament amb l'\emph{Open Source} s'exposen els arguments i s'afegeix cert
to d'enveja que per altra banda crec innecessari i menys d'una persona que està acostumada a viure a contra corrent
sense rebre gaire reconeixement social.

El llibre en si està prou bé és molt complet però pel meu gust un xic repetitiu. Tal vegada s'en podria
fer una versió mes reduïda amb un subconjunt dels assajos que apareixen i se n'obtindria més rendiment,
seria més ràpid i mes fàcil de difondre al públic en general. No pretenc desmerèixer l'autor
però he de reconèixer que tot hi estar molt interessant en el tema alguns dels capítols
s'han fet difícils de llegir tot i ser breus. Si bé és cert que eren del final quan ja s'ha exposat
gairebé tot i simplement és recargolen més els arguments.

   \let\thefootnote\relax\footnotetext{
 Aquest document està baix llicència \href{http://creativecommons.org/licenses/by-sa/3.0/}{Creative Commons Atributive Share-Alike 3.0}
 per tant es pot compartir, modificar i distribuir, però citant l'autor original i sense modificar la llicència.\bigskip}
\let\thefootnote\relax\footnotetext{El document en versió digital i el codi font el trobareu a \\
\url{https://github.com/bmiro/fsfs}\bigskip}
\let\thefootnote\relax\footnotetext{Aquest document ha estat desenvolupat emprant programari lliure:}
\let\thefootnote\relax\footnotetext{\href{http://www.tug.org/applications/pdftex/}{\LaTeX} i \href{http://www.tug.org/applications/pdftex/}{Kile} per el text.
}

\let\thefootnote\relax\footnotetext{
\begin{center}
\begin{tabular}{cc}
\includegraphics[height=35pt,keepaspectratio=true]{diagrames/by-sa.png}
 & \includegraphics[height=35pt,keepaspectratio=true]{diagrames/gnu.png}
\end{tabular}
\end{center}
\begin{center}
\begin{tabular}{cc}
 \includegraphics[height=35pt,keepaspectratio=true]{diagrames/latex.png}
 & \includegraphics[height=35pt,keepaspectratio=true]{diagrames/kile.png}
\end{tabular}
\end{center}
}


\end{document}




